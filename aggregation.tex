\documentclass[a5paper]{jsarticle}


% 数式
\usepackage{amsmath,amsfonts}
\usepackage{bm}
\usepackage[italicdiff]{physics}
% 画像
\usepackage[dvipdfmx]{graphicx}


\begin{document}

\title{数理モデルのアグリゲーション}
\author{木津川楓輝}
\date{}
\pagestyle{empty}
\maketitle
\section{はじめに}
自然現象の本質を数式によって表現したものを数理モデルと言います。
自然科学ではしばしば数理モデルが解析されますが、同じ現象に対するモデルであっても、より単純なモデルの方が直観的に理解しやすいでしょう。
また、複雑なモデルは、データによって決めなければならない未知パラメータの数が多いため、挙動が不安定になる傾向があります。
そのため、多くの変数を含む複雑なモデルから、変数を束ねてより少ない変数で表されるモデルを作る場合があり、このような操作をアグリゲーションと言います。

変数の数を減らすと確かにシンプルなモデルになりますが、元のモデルに対して誤差が発生しそうに思えるかもしれません。
今回は、アグリゲーションを行ったときに誤差を引き起こさない場合があるのかといった問題について考えてみたいと思います。
\section{完全アグリゲーション}
多くの変数からなる複雑なモデルにおいて、系の時間発展が
\[
  \dv{x_i}{t} = f_i(x_1, x_2, \cdots, x_n) \qquad i = 1, 2, \cdots, n
\]
のような微分方程式で記述されるとしましょう。
このモデルは、$t = 0$における系の状態をもとにして、将来の状態を予測するために用いられるものとします。
ここで、我々が本当に知りたい量は$x_i$の全てではなく、それらを用いて計算される少数のマクロ変数、
\[
  y_j = g_j(x_1, x_2, \cdots, x_n) \qquad j = 1, 2, \cdots, m
\]
であったとします($m < n$)。
ここで、これら
\section{ロトカ・ボルテラ競争系}
ここまで抽象的な話が続いてきたので、アグリゲーションの具体例を見るために、まずロトカ・ボルテラ競争系というモデルをご紹介します。
これは、競争関係にある2種の生物の個体数がどのように時間変化するかを表すモデルで、2種の個体数を$x$と$y$で表すと、それらの時間変化は
\begin{eqnarray*}
  \dv{x}{t} & = & r_1x\left(1 - \frac{x + ay}{K_1}\right) \\
  \dv{y}{t} & = & r_2y\left(1 - \frac{bx + y}{K_2}\right)
\end{eqnarray*}
で表されるというものです。
\end{document}