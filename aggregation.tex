\documentclass[a5paper]{jsarticle}


% 数式
\usepackage{amsmath,amsfonts}
\usepackage{bm}
\usepackage[italicdiff]{physics}
% 画像
\usepackage[dvipdfmx]{graphicx}


\begin{document}

\title{数理モデルのアグリゲーション}
\author{木津川楓輝}
\date{}
\pagestyle{empty}
\maketitle
\section{はじめに}
自然現象の本質を数式によって表現したものを数理モデルと言います.
自然科学ではしばしば数理モデルが解析されますが,対象に関する情報をできる限り多く詰め込んだ詳細なモデルは複雑になる傾向があります.
同じ現象に対するモデルであっても,より単純なモデルの方が直観的に理解しやすいでしょう.
また,複雑なモデルは,データによって決めなければならない未知パラメータの数が多いため,挙動が不安定になる傾向があります.
そのため,多くの変数を含む複雑なモデルから,変数を束ねてより少ない変数で表されるモデルを作る場合があり,このような操作をアグリゲーションと言います.

変数の数を減らすと確かにシンプルなモデルになりますが,元のモデルに対して誤差が発生しそうに思えるかもしれません.
今回は,アグリゲーションを行ったときに誤差を引き起こさない場合があるのかといった問題について考えてみたいと思います.
\section{完全アグリゲーション}
多くの変数からなる詳細なモデルにおいて,系の時間発展が
\begin{equation}
  \dv{x_i}{t} = f_i(x_1, x_2, \cdots, x_n) \qquad i = 1, 2, \cdots, n \label{detail}
\end{equation}
のような微分方程式で記述されるとしましょう.
このモデルは,$t = 0$における系の状態をもとにして,将来の状態を予測するために用いられるものとします.
ここで,我々が本当に知りたい量は$x_i$の全てではなく,それらを用いて計算される少数のマクロ変数,
\begin{equation}
  y_j = g_j(x_1, x_2, \cdots, x_n) \qquad j = 1, 2, \cdots, m \label{y}
\end{equation}
であったとします($m < n$).
ここで,これらのマクロ変数のみからなる単純なモデルが
\begin{equation}
  \dv{y_j}{t} = F_j(y_1, y_2, \cdots, y_m) \label{simple}
\end{equation}
で表され,このモデルによる予測値$y_j(t)$と,詳細なモデル(\ref{detail})による$x_i(t)$から(\ref{y})を用いて計算したものが同一の結果を与えるとき,単純なモデル(\ref{simple})を詳細なモデル(\ref{detail})の代わりに用いることができます.
このとき,完全アグリゲーションが成立していると言います.
\section{ロトカ・ボルテラ競争系}
ここまで抽象的な話が続いてきたので,アグリゲーションの具体例を見るために,まずロトカ・ボルテラ競争系というモデルをご紹介します.
これは,競争関係にある2種の生物の個体数がどのように時間変化するかを表すモデルで,2種の個体数を$x$と$y$で表すと,それらの時間変化は
\begin{eqnarray}
  \dv{x}{t} & = & r_1x\left(1 - \frac{x + ay}{K_1}\right) \nonumber\\[-3mm]
  \label{lv} && \\[-3mm]
  \dv{y}{t} & = & r_2y\left(1 - \frac{bx + y}{K_2}\right) \nonumber
\end{eqnarray}
で表されるというものです.
基本的には各種の個体数が多いほど増殖率も大きくなるのですが,このモデルにはそれだけではなく,同一種の個体数がある程度増加すると増殖率が低下する効果,および種間競争によって増殖率が減少する効果が含まれています.
第1種の増殖率$\dv{x}{t}$を例にとると,右辺の$r_1x$の項により,第1種の個体数が多いほど増殖率が大きくなる効果が表されています.
また,$-\frac{r_1x^2}{K_1}$の項は,第1種の個体数がある程度増加すると第1種の増殖率が低下する効果を表し,$-\frac{r_1axy}{K_1}$の項は種間競争の効果を表します.
種間競争による第2種の1個体の影響を第1種の1個体分の影響に換算する係数が$a$です.
例えば,種によって用いる資源が異なれば,同種の1個体よりも影響が少ないので$a < 1$となるでしょう.
しかし,2種が共通の資源を要求し,第2種が第1種よりも多くの資源を消費するなら,$a > 1$となるでしょう.
同様に,$b$は第2種の増殖率に対して第1種の1個体が与える影響を,第2種の1個体分に換算する係数です.

このモデルを解析すると,$K_1, K_2, a, b$の関係によって結果が変わり,1種のみが勝ち残ったり2種が安定に共存したりすることが分かります.
今回は詳しく扱いませんが,興味のある方は勉強してみて下さい.
\section{多数種のロトカ・ボルテラ競争系}
先ほど紹介したロトカ・ボルテラ競争系(\ref{lv})を多数種に拡張すると,次のようになります.
\begin{equation}
  \dv{x_i}{t} = r_ix_i\left(1 - \sum_j\frac{\alpha_{ij}x_j}{K_i}\right) \qquad i = 1, 2, \cdots, n \label{multi}
\end{equation}
ここで,$\alpha_{ij}$は第$j$種が第$i$種に及ぼす影響を第$i$種の1個体分に換算する係数です(したがって,$\alpha_{ii} = 1$が成り立ちます).
\section{ロトカ・ボルテラ競争系のアグリゲーション}
前節ではロトカ・ボルテラ競争系を多数種に拡張しましたが,例として3種の場合((\ref{multi})において$n = 3$とした場合)を考えてみましょう.
ここで,第1種の動態にだけ興味があるとして,残りの2種を纏めて「第1種にとっての競争者の量」という1変数で表すことはできないでしょうか.
すなわち,
\[
  y_1 = x_1 \qquad y_2 = x_2 + x_3
\]
として,$y_1$と$y_2$に関するモデルを作って,元の3種のモデルによる予測と一致するようにできる条件を考えます.
計算してみると(かなり大変ですが),
\[
  r_2 = r_3, K_2 = K_3, \alpha_{12} = \alpha_{13}, \alpha_{21} = \alpha_{31}, \frac{\alpha_{23} + \alpha_{32}}{2} = 1
\]
が必要十分条件であることが分かります.
これらの条件は,第2種と第3種がよく似ている場合は束ねても予測の結果が変わらないことを表しており,直観的にも理解しやすいでしょう.
ただし,最後の条件から分かるように,束ねられる2種が完全に同一である必要はありません.
\section{終わりに}
いかがだったでしょうか.
今回は,アグリゲーションにおいて誤差が発生しない条件について考えましたが,この条件はかなり厳しいため,誤差を最小にするような単純化モデルについての研究も行われています.
他にも数理モデルに関する面白い話題は沢山あるので,興味を持った方は勉強してみて下さい.
最後まで読んで頂きありがとうございました.
\section{参考文献}
巌佐庸『数理生物学入門―生物社会のダイナミックスを探る』共立出版,1998
\end{document}